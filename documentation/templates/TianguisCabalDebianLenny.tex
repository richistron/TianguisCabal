\documentclass[a4paper,10pt]{article}
\usepackage[utf8]{inputenc}
\usepackage[spanish]{babel}

\newenvironment{mylisting}
{\begin{list}{}{\setlength{\leftmargin}{1em}}\item\scriptsize\bfseries}
{\end{list}}

\newenvironment{mytinylisting}
{\begin{list}{}{\setlength{\leftmargin}{1em}}\item\tiny\bfseries}
{\end{list}}

\hyphenation{vi-si-ble}

%opening
\title{Configuración del TianguisCabal}
\author{Carlos A. Gonzalez}

\begin{document}

\maketitle

\begin{abstract}
El siguiente documento explica como configurar el proyecto TianguisCabal en Debian GNU/Linux, en Ubuntu y demás distribuciones basadas en Debian GNU/Linux. Al finalizar la instalación, el proyecto estará visible en \texttt{http://localhost/tianguiscabal}. 
\end{abstract}

\section{Configurando niveles de error en PHP}

En en el archivo \texttt{/etc/php5/apache2/php.ini}, busquen \texttt{error\_reporting}. Así  es como esta en debian:

\begin{mylisting} \begin{verbatim}
error_reporting = E_ALL & ~E_NOTICE
\end{verbatim} \end{mylisting}

Y así es como debe de quedar esta linea con los cambios sugeridos:

\begin{mylisting} \begin{verbatim}
error_reporting = E_ALL |  E_STRICT
\end{verbatim} \end{mylisting}

\section{Configurando niveles de error en Apache}

En en el archivo \texttt{/etc/apache2/apache2.conf}, busquen \texttt{LogLevel}. Así es como esta en debian:

\begin{mylisting} \begin{verbatim}
LogLevel warn
\end{verbatim} \end{mylisting}

Y así es como debe de quedar esta linea con los cambios sugeridos:

\begin{mylisting} \begin{verbatim}
LogLevel debug
\end{verbatim} \end{mylisting}

Los errores que tenga apache, se verán en el archivo \texttt{/var/log/apache2/error.log}.


\end{document}